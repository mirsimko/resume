%%%%%%%%%%%%%%%%% PREAMBLE %%%%%%%%%%%%%%%%%%%%%%%%%%%%
%Change the font size of your document - 10pt, 12.1pt, etc.
\documentclass[a4paper,11pt,oneside]{article}
\usepackage[utf8]{inputenc}
\usepackage{setspace}
\usepackage[hidelinks]{hyperref}

\usepackage{graphicx}
\graphicspath{{images/}} %upload your signature to this file
%Change the margins to fit your CV/resume content
\usepackage[left=0.6in, right=1in, bottom=0.8in, top=1in]{geometry}

%Skype information - include your Skype name for a link to add you on Skype
\newcommand*{\Skype}{\href{skype:john.smith?add}{john.smith}} 
\newcommand{\Absender}[1][\normalsize]{\Skype} 

%Changes the page numbers - {arabic}=arabic numerals, {gobble}=no page numbers, {roman}=Roman numerals
\pagenumbering{gobble}

%%%%%%%%%%%%%%%%% END OF PREAMBLE %%%%%%%%%%%%%%%%%%%%%

\begin{document}

%%%%%%%%%%%%%%%%% NAME OF APPLICANT %%%%%%%%%%%%%%%%%%%

\noindent  \LARGE{\textbf{Miroslav Šimko, PhD}}  \\
\vspace{-2ex}
% \hline 
\normalsize

%%%%%%%%%%%%%%%%% CONTACT INFORMATION %%%%%%%%%%%%%%%%%
% Your email address, website, and Skype name are links to send email, open your website and add you on Skype. 

\begin{center}
\begin{tabular}{l l}
 Computer vision / Python / C++ developer & \hspace{1in} \href{mailto:ms@iolabs.ch}{ms@iolabs.ch} \\
 at ioLabs AG       & \hspace{1in} Phone: +420 605 063 354, +81 70-9020-3002 \\
  \url{http://iolabs.ch} & \\
  Currently based in Japan
\end{tabular}
\end{center}

\vspace{1em}

%%%%%%%%%%%%%%%%% MAIN BODY %%%%%%%%%%%%%%%%%%%%%%%%%%%
% The main body is contained in a tabular environment. To move sections onto the next page, simply end the tabular environment and begin a new tabular environment.

\noindent \begin{tabular}{@{} l l}
\hline \\
 \Large{Education}   
     & \textbf{2013--2021, Ph.D.}, Experimental Nuclear and Particle Physics \\
     & {Czech Technical University} \\
     & {Faculty of Nuclear Sciences and Physical Engineering} \\
     & \textbf{Thesis:} Study of Heavy Flavor at the STAR Experiment \\[.2cm]
%      & \\
     & \textbf{2011--2013, M.Sc.}, Experimental Nuclear and Particle Physics \\
     & {Czech Technical University} \\
     & {Faculty of Nuclear Sciences and Physical Engineering} \\
     & \textbf{Thesis:} Design and Optimalization of the Optical Readout System \\
     & for Electromagnetic Calorimeter FOCAL for the ALICE Experiment \\[.2cm]
%      & \\
     & \textbf{2008--2011, B.Sc.}, Experimental Nuclear and Particle Physics \\
     & {Czech Technical University} \\
     & {Faculty of Nuclear Sciences and Physical Engineering} \\
     & \textbf{Thesis:} Detector Control System for the ALICE Experiment \\
     \\
   
  \Large{Experience} & \textbf{2022--present: ioLabs AG: 3D-computer vision on meshes} Trained model \\
     & for 3D object recognition and segmentation of 3D models of buildings. \\
     & Written in Python using the Pytorch framework and Mesh-CNN model.\\[.2cm]
     & \textbf{2021--present: ioLabs AG: Computer-vision object detection on technical} \\ 
     & \textbf{documents} Developed object detection and automatic placement of a stamp \\
     & in documents, including automatic orientation detection. Used Pytorch  \\
     & and Mask-RCNN\@. The model was pruned and quantized for deployment.\\[.2cm]
     & \textbf{2019--2020: ioLabs AG: Geometry engine for collision detection}, written\\
     &  in C++ and Python, using Open CASCADE and VTK\\[.2cm]
     & \textbf{2014--2021: Analysis}: Reconstruction of the $\Lambda_\mathrm{c}$ baryon  at the STAR \\
     & experiment Brookhaven National Laboratory, USA; Collaboration between \\
     &  Lawrence Berkeley National Laboratory, USA, and Czech Technical University;\\
     & Analysis, using ``big-data'' techniques on computing clusters; \\
     & Code written in C++ and Root, using machine learning from the TMVA package \\
     & (Boosted-Decision Trees)\\[.2cm]
     & \textbf{2015--2019: STAR Zero-Degree-Calorimeter on-call expert} at Brookhaven \\ 
     & National Laboratory, USA\@; Responsible for calibration, checks, maintenance, \\
     & and upgrades of crucial detetector components; Calibration code written \\
     & in C++ and Root \\[.2cm]
     & \textbf{2013--2014: Lawrence Berkeley National Laboratory, USA\@: Simulations } \\ 
     & \textbf{for the Pixel sensors} at the STAR experiment; Written in C++ and Root\\[.2cm]
     & \textbf{2011--2013: Detector-Control-System expert for the Silicon-Drift Detector} \\
     & for the ALICE experiment, LHC, CERN, Switzerland; Responsible for maintenance \\
     & and smooth operation of the detector, written upgrades to the system in PVSS. \\
    

\end{tabular}

\newpage

%%%%%%%%%%%%%%%%% REFERENCES %%%%%%%%%%%%%%%%%%%%%%%%%%
% The reference section has links to your references' websites and email addresses.
\noindent  \begin{tabular}{@{} l l}
  \Large{Languages}   & Czech/Slovak (native), English (fluent), Japanese (advanced intermediate), \\
\Large{and Skills}    & French (intermediate)  \\[.2cm]
     & Statistical analysis, machine learning, programming for computing clusters, \\
     & Docker, Jenkins, git\\[.2cm]
     & Programming in C, C++, Python, Root, Pytorch, BASH,  \\
     & PVSS, National Instruments LabView \\[.2cm]
     & Driver's license B\\[.3cm]
  \Large{Teaching}  & \textbf{Czech Technical University} \\
     & \textbf{Faculty of Nuclear Sciences and Physical Engineering} \\
     & Student Physics Laboratory Practice, 2014--2019\\
     \\
\Large{Interests}    
     & Physics, informatics, photography, hiking, sport (bicycle, ski, canoeing), literature \\
     \\
\Large{Paper}
     & J.\ Adam et al., \emph{``First Measurement of $\Lambda_\mathrm{c}$ Baryon Production in Au+Au Collisions} \\
     & \emph{at $s_\mathrm{NN} = 200\,$GeV''}, Phys.\ Rev.\ Lett.\ \textbf{124}, 172301 (2020) \\
     \\
\Large{Conferences}
     & \textbf{Quark Matter 2018, Venice, Italy} May 13--19, 2018 \\
     & Poster: Measurement of $\mathrm{\overline{\Lambda_c}^-/\Lambda_c^+}$ ratio in Au+Au collisions at $\sqrt{s_\mathrm{NN}} = 200\,$GeV\\
     &with the STAR experiment \\[.2cm]
     & \textbf{3-Kings Conference 2018, Košice, Slovakia} Jan 5, 2018 \\
     & Talk: Measurement of open charm in relativistic-heavy-ion collisions \\[.2cm]
     & \textbf{19th Conference of Czech and Slovak Physicists, Prešov, Slovakia}\\
     & Sep 4--7, 2017 \\
     & Talk: Measurement of the $\Lambda_\mathrm{c}$ baryon at $\sqrt{s_\mathrm{NN}} = 200\,$GeV with the STAR \\
     & experiment\\[.2cm]
     & \textbf{EPS Conference on High Energy Physics, Venice, Italy} Jul 5--12, 2017\\
     & Talk: Measurements of open charm hadron production in Au+Au collisions by  \\
     & the STAR experiment \\[.2cm]
     & \textbf{Hot Quarks, South Padre Island, Texas, USA} Sep 12--17, 2016\\
     & Talk: Measurements of open charm hadrons at the STAR experiment \\[.2cm]
     & \textbf{Quark Matter 2015, Kobe, Japan} Sep 27--Oct 3, 2015\\
     & Poster: $\Lambda_\mathrm{c}$ baryon production at $\sqrt{s_\mathrm{NN}} = 200\,$GeV \\[.2cm]
     & \textbf{ICPAQGP2015, Kolkata, India} Feb 2--6, 2015\\
     & Talk: Heavy Flavor Tracker at the STAR Experiment \\[.2cm]
     & \textbf{18th Conference of Czech and Slovak Physicists, Olomouc, Czech} \\
     & \textbf{Republic} Sep 16--19 2014 \\
     & Talk: Simulations for the HFT--Pixel detector at the STAR experiment\\[.2cm]
     & \textbf{Workshop VERTEX2014, Mácha Lake, Czech Republic} Sep 15--19, 2014\\
     & Poster: Simulations for the HFT--Pixel detector at the STAR experiment \\
     & \\
\end{tabular}

\noindent \begin{tabular}{@{} l l}
  \Large{References} & \small \textbf{Github page:} \\
  & \small \url{https://github.com/mirsimko} \\[.1cm]
  & \small \textbf{PhD-analysis code} \\
  & \small \url{https://github.com/mirsimko/auau200GeVLambdaCAna} \\[.1cm]
  & \small \textbf{ioLabs AG} \\
  & \small \url{https://iolabs.ch}
\end{tabular}

\end{document}

